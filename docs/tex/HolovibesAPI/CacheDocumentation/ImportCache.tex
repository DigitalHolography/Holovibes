%\begin{ImportCache}

\subsection{ImportCache}
\begin{itemize}
    \item ImportType
    \item ImportFrameDescriptor
    \item ImportFilePath
    \item LoadFileInGpu
    \item StartFrame
    \item EndFrame
    \item FileNumberOfFrame
    \item LoopFile
    \item InputFps
    \item CurrentCameraKind
\end{itemize}

\subsubsection{ImportType}
\noindent
Type : Enum class ImportTypeEnum \{ None, File, Camera \}\\
Unit : struct
DefaultValue : None\\
Pre-Condition : None\\
Description : One the most important variable in Holovibes. It allow to run or not Holovibes.\\
\begin{itemize}
    \item None : Holovibes does not run and stop all workers if any were created. 
    \item File : Holovibes run using the data stream from the file 
    \item Camera : Holovibes run using the data stream from the configured camera 
\end{itemize}

\subsubsection{ImportFrameDescriptor}
\noindent
% Maybe Frame descritor could use uint16_t/uint32_t instead of short and int, we are in 2023... short are bad%
Type : class FrameDescriptor \{uint16\_t : width, uint16\_t : height, uint32\_t : depth, enum Endianness \{ LittleEndian, BigEndian \} byteEndian \} \\
Unit : struct
DefaultValue : { 0, 0, 0, LittleEndian }\\
Pre-Condition : None\\
Description : It is automatically set when you configure a camera or you load a file. Hence, it does not need to be edited manually.

\subsubsection{ImportFilePath}
\noindent
Type : String\\
Unit : Path\\
DefaultValue : ""\\
Pre-Condition : None\\
Description : Set the file to load. When this variable is changed, the new file at this path is loaded and all variables contained in the footer are loaded inside the Microcaches.

\subsubsection{LoadFileInGpu}
\noindent
Type : Boolean \\
Unit : None\\
DefaultValue : true \\
Pre-Condition : None\\
Description : When true, this variable indicates that the file to process has to be uploaded entirely in the GPU or split before beginning to send to the GPU. Set to false when the file is too large for GPU VRAM. If it's false then you can change the split size with the variable FileBufferSize in the AdvanceCache.

\subsubsection{StartFrame}
\noindent
Type : Integer \\
Unit : Frame index (Begin at 1). \\
DefaultValue : 1 \\
Pre-Condition : greater than 0 and lesser than FileNumberOfFrame\\
Description : Set the start index when reading through and accessing the input file\\

\subsubsection{EndFrame}
\noindent
Type : Integer\\
Unit : Frame index (Begin at 1)\\
DefaultValue : 1\\
Pre-Condition : greater than 0 and lesser than FileNumberOfFrame\\
Description : Set the end index reading through and accessing the input file\\

\subsubsection{FileNumberOfFrame}
\noindent
Type : Integer\\
Unit : Number of frames\\
DefaultValue : 1\\
Pre-Condition : greater than 0\\
Description : Total number of frames of the file. You shouldn not change this in normal case because it is automatically changed loading a file(by changing the ImportFilePath variable)\\

\subsubsection{LoopFile}
\noindent
Type : Boolean\\
Unit : None\\
DefaultValue : true\\
Pre-Condition : None\\
Description : Holovibes will either stop after reading the file until EndFrame or loop on it and start with the frame on index StartFrame\\

\subsubsection{InputFps}
\noindent
Type : Integer\\
Unit : Number of frames per seconds (fps)\\
DefaultValue : 60\\
Pre-Condition : 0\\
Description : Set the speed at which the file is read. Increasing it increases the computation power needed. Decreasing it will decrease the output fps\\

\subsubsection{CurrentCameraKind}
\noindent
Type : enum CameraKind \{ None or Adimec or IDS or Phantom or BitflowCyton or Hamamatsu or xiQ or xiB or OpenCV\}\\
Unit : enum \\
DefaultValue : None\\
Pre-Condition : None\\
Description : If different to None, Holovibes will edit ImportType variable to Camera and start the computation\\

%\end{ImportCache}