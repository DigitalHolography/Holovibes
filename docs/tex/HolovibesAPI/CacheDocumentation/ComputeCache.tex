%\begin{ComputeCache}

\subsection{ComputeCache}
\begin{itemize}
    \item ComputeMode
    \item ImageType
    \item BatchSize
    \item TimeStride
    \item Filter2D
    \item SpaceTransformation
    \item TimeTransformation
    \item TimeTransformationSize
    \item Lambda
    \item ZDistance
    \item Convolution
    \item PixelSize
    \item UnwrapHistorySize
    \item Unwrap2DRequested
    \item TimeTransformationCutsEnable
\end{itemize}

\subsubsection{ComputeMode}
\noindent
Type : Enum class ComputeModeEnum \{ Raw, Hologram \} \\
Unit : Input processes \\
Defaul-tValue : Raw \\
Pre-Condition : None\\
Description : Describe the behaviour of Holovibes. On raw no processing are done.\\

\subsubsection{ImageType}
\noindent
Type : Enum class ImageTypeEnum \{ Modulus, SquaredModulus, Argument, PhaseIncrease, Composite \}\\
Unit : \\
DefaultValue : Modulus\\
Pre-Condition : None\\
Description : Each ImageType provides access to some specific processing. Composite give access to RGB and HSV coloring. PhaseIncrease and Argument change the phase value (For all complex pixel c : atan(Im(c)/Re(c) and conjuguate of the phase between previous and current frame respectively)\\

\subsubsection{BatchSize}
\noindent
Type : Integer.\\
Unit : Number of frames.\\
DefaultValue : 1\\
Pre-Condition : None\\
Description : Sets the number of consecutive input frames onto which the spatial transformation is computed at once.\\

\subsubsection{TimeStride}
\noindent
Type : Integer.\\
Unit : Number of frames.\\
DefaultValue : 1\\
Pre-Condition : None\\
Description : Sets the number of input frame lag between two consecutive time transformations applied to the current data block.\\

\subsubsection{Filter2D}
\noindent
Type : Structure Filter2DStruct \{ bool : enabled, int : inner\_radius, int : outer\_radius\}\\
Unit : squared meters\\
DefaultValue : \{ false, 0, 1\}\\
Pre-Condition : None\\
Description : Apply a filter between main frames and the surface between squares of inner\_radius and outer\_radius size.\\

\subsubsection{SpaceTransformation}
\noindent
Type : Enum SpaceTransformationEnum \{ NONE, FRESNELTR, ANGULARSP \}\\
Unit : calcul\\
DefaultValue : NONE\\
Pre-Condition : None\\
Description : FRESNELTR will apply one fft on batch\_size frames and apply a quadratic mask. ANGULARSP will apply one forward fft then a spectral mask and a reverse fft on batch\_size frames.\\

\subsubsection{TimeTransformation}
\noindent
Type : Enum TimeTransformationEnum \{ STFT, PCA, NONE, SSA\_FFT\}\\
Unit : calcul\\
DefaultValue : STFT\\
Pre-Condition : NONE\\
Description : The different TimeTransformation stands respectively for Shot-time Fourrier Transformation, Principal Component Analysis, no transformation, Self-adaptative Spectrum Analysis Short-time Fourrier Transformation. All are applied on time\_transformation\_size\\

\subsubsection{TimeTransformationSize}
\noindent
Type : Integer\\
Unit : Number of images\\
DefaultValue : 1\\
Pre-Condition : greater than 0\\
Description : Sets the number of input frames forming the data block onto which the time transformation is applied\\

\subsubsection{Lambda}
\noindent
Type : Float\\
Unit : nano-meters\\
DefaultValue : 852\\
Pre-Condition : None\\
Description : Optical radiation wavelength\\

\subsubsection{ZDistance}
\noindent
Type : Float\\
Unit : meters\\
DefaultValue : 1.5\\
Pre-Condition : None\\
Description : Hologram rendering/unfocus distance\\

\subsubsection{Convolution}
\noindent
Type : Struct ConvolutionStruct \{ bool : enabled, std::string : type, bool : divide, std::vector<float> : matrix \}\\
Unit : struct\\
DefaultValue : \{ false, "None", divide, []\}\\
Pre-Condition : None\\
Description : Bidimensional convolution of the current displayed image by a user-defined kernel.\\

\subsubsection{PixelSize}
\noindent
Type : Float\\
Unit : micro-meters\\
DefaultValue : 12\\
Pre-Condition : None\\
Description : Pixel pitch is the distance between center of adjacent pixels\\

\subsubsection{UnwrapHistorySize}
\noindent
Type : Integer\\
Unit : \\
DefaultValue : 1\\
Pre-Condition : None\\
Description : \\

\subsubsection{Unwrap2DRequested}
\noindent
Type :Boolean \\
Unit : \\
DefaultValue : false\\
Pre-Condition : None\\
Description : \\

\subsubsection{TimeTransformationCutsEnable}
\noindent
Type : Boolean\\
Unit : \\
DefaultValue : false\\
Pre-Condition : None\\
Description : \\

%\end{ComputeCache}