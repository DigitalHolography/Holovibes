Digital holography refers to all the techniques for acquiring and processing holograms from a digital sensor. It provides a versatile approach for capturing optical phase data, commonly yielding three-dimensional surface profiles and optical thickness images. Numerous recording and processing techniques have been devised to, for example, analyse optical wave attributes, encompassing amplitude, phase, and polarisation states. This collective capability renders digital holography an exceptionally potent tool for applications in ophthalmology. \\

These techniques herald a new global transition to digital coherent light imaging with camera sensor arrays by application of radar processing methods. This allows digital image rendering \cite{shimobaba2012computational} and digital wave-field manipulation \cite{kumar2013subaperture, javidi2021roadmap} after signal acquisition. \\

Ultra-fast cameras, Graphics Processing Unit (GPU) and parallel computing power have  been standardised over the past decade. This permits the development of non-invasive digital holographic techniques that may lead to the democratisation of compute-intensive imaging in real-time \cite{Leutenegger2011Real, samson2011video, Bencteux2015Holographic, Puyo2020Realtime}.\\

In the case of ophthalmology, offline ultrahigh-speed digital holography already permits a high-quality non-invasive structural and functional imaging \cite{ Hillmann2016, Puyo2018Vivo, tomczewski2022light} of the human eye.\\

Holovibes is a free software dedicated to the calculation of digital holograms in real-time. It has been designed to accelerate the transition to high throughput digital optics, creating holograms from interferograms. Input data can be grabbed from a digital camera or loaded from files recorded beforehand. Massive amounts of data can be handled robustly at high throughput, saved to disk, and visualised in real-time without any risk of frame dropping thanks to the use of several configurable input and output memory buffers. This software is developed in C++ and CUDA on Windows 7 (64-bit) and Windows 10 (64-bit) operating systems. It uses NVIDIA graphics processing units (GPU) for parallel computation.

%In this report, we describe the software architecture that we use for high-speed rendering of digital holograms. In Section \ref{sec:progarchi}, we give a brief overview of the key components of our program while Section \ref{sec:datastreams} is dedicated to the computation workflow and our data structures. Then, in Section \ref{sec:progopti}, we detail the various optimisations implemented to speed up the processing time and increase the input throughput by taking advantage of parallel computations. Finally, in Section  \ref{sec:benchmarks}, we demonstrate real-time and sustained high throughput holographic imaging from a camera input stream of 8-bit 256-by-256 interferogram frames at a rate of $\sim$60,000 frames per second ($\sim$4 GB/s), with a state-of-the-art streaming camera, computer and commodity GPU. To the best of our knowledge, this software implements the fastest real-time and low-latency rendering engine reported for the computation of digital holograms from a high throughput stream of interferograms.

%% Non! On n'annonce pas le plan du document dans un paragraphe comme un commentaire de document d'histoire-geographie de collégien. Les titres doivent parler d'eux-même, la table des matière c'est pas pour les chiens %%