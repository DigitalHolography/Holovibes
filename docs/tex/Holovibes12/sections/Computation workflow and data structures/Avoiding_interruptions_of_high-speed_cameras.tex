To read frames from high-speed cameras we rely on frame grabbers that can be configured to let the CPU acquire raw frames without having to process end-of-frame interrupts. These interruptions can become a bottleneck when dealing with very high frame rates. The program tracks the number of interrupts emitted by the camera since the last acquisition to determine the number of frames that are ready to be acquired. Images are extracted directly from the camera buffer and additional metadata is neither emitted nor read. Furthermore, we rely on a pinned memory buffer allocated on the host to allow fast copies from the CPU memory to the GPU memory. \\

Pinned memory is used as a staging area for transfers from the device to the host. It increases the data throughput and reduces potential lags by avoiding the cost of the transfer between page-able and pinned host arrays by directly allocating our host arrays in pinned memory. Allocates pinned host memory in CUDA C/C++ using \emph{cudaHostAlloc()} that allows use of flags, and de-allocates it with \emph{cudaFreeHost()}