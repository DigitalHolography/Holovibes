The CPU and the GPU are assigned to different tasks which are dissociated and run asynchronously. However, at some point, the host needs the result of the processing performed on the device. Thus, synchronisation is necessary as the CPU has to wait before retrieving data from the GPU. It is fundamental to correctly handle the synchronisation as too many synchronisations would have a negative impact on the performance and not enough synchronisations could corrupt the data. In our case, only the output rendering requires synchronisation between the CPU and the GPU. We reduced the number of calls to the \textit{cudaDeviceSynchronize} method to one at each output frame computed. The rest of the time both processing units run fully asynchronously.