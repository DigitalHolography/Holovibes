The main component of the program is the computation pipeline which aims to streamline operations for image rendering. The pipe is an array of functions that are applied sequentially and repeatedly to the input data stream from a file or camera. This allows for an optimised way to execute arbitrary computations a large number of times as it reduces the number of conditional branches (if/else) encountered during each loop iteration. In practice, the type of numerical computations applied continuously to the data stream consists of a set of parameterised deterministic functions. As this set of functions is rarely modified by the user, it makes sense to create the pipe once when the program starts and update its content only when a setting is changed through the user interface. These settings include hologram reconstruction distance, choice of rendering algorithm, short-time analysis window size for temporal signal demodulation, or post-processing options.