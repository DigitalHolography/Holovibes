After being converted to single-precision complex-valued data arrays, each frame batch undergoes a two-step parallel processing: (1) a spatial transform to create a hologram from each recorded interferogram ; (2) a temporal transform applied to the batch of holograms.\\

When the input queue contains one full batch of frames, input data is converted to complex representations and copied to the space transformation buffer. This buffer is used to apply a 2D spatial Fast Fourier Transform (FFT) to the extracted batch. This operation is applied in parallel to each frame of the batch with \textit{cuFFT} using \textit{XtPlanMany} data shaping. Applying the transformation to a batch of images allows to reduce the number of memory copies and increase the processing speed.\\

Then, the batch of frames is inserted into the time transformation queue. This data structure can contain a number of frames that is a multiple of batch size. Once the time transformation queue is full, the temporal transform of the batch of holograms is done by Principal Component Analysis (PCA), which is achieved in practice by data eigen-decomposition. In particular, the construction of a covariance matrix of a batch of holograms, its eigen-decomposition, and the projection of the result into a sub-basis are performed with \textit{cublasCgemm} and \textit{cusolverDnCheevd}. Then, data is copied to an accumulation buffer of the same size, and frames between two user-selected indices are flattened into a single image.