\begin{figure*}[t]
    \centering
    \includegraphics[width=\linewidth]{figures/holo_article.pdf}
    \caption{Diagram of the general computation workflow for high-speed rendering of digital holograms. The input frames (8 or 16-bit unsigned integer-valued images). Space and time transformations are applied sequentially to batches of frames to produce demodulated, complex-valued (one 32-bit single float per quadrature) holograms, which are flattened to a real-valued reconstructed image (one single float per pixel). The continuous stream of input and output frames is buffered to prevent any frame drop.
    }
    \label{fig:computation_workflow}
\end{figure*}

In this report, we choose to focus on the implementation details rather than diving into the details of hologram rendering. The basic computation workflow is depicted in Figure \ref{fig:computation_workflow}. Our data structures were created to meet two key requirements: (1) avoid any frame drop in the continuous stream of input images ; (2) enable more efficient parallelisation of the computations used for image rendering. To this effect, we make extensive use of buffers (fixed size contiguous memory area) and queues (ring buffer: fixed-size buffer connected end-to-end) that are allocated in the GPU memory. These structures are designed to be manipulated in a thread-safe fashion without causing a race condition (when the program's behaviour is determined by uncontrollable events such as thread execution order).

\subsection{Input}
The input queue is a "ring" memory structure located on the GPU. It aims to act as a staging area between the raw image stream from the camera, at a throughput determined by the frame grabber and the computation pipeline. It is a thread-safe producer/consumer queue as one thread only writes data to it while another only reads data from it. The input queue is the most critical component of the program because it is the entry point of all frames and it is being accessed by two threads simultaneously. % explain ring buffer

\subsection{Space and time transformations}
After being converted to single-precision complex-valued data arrays, each frame batch undergoes a two-step parallel processing: (1) a spatial transform to create a hologram from each recorded interferogram ; (2) a temporal transform applied to the batch of holograms.\\

When the input queue contains one full batch of frames, input data is converted to complex representations and copied to the space transformation buffer. This buffer is used to apply a 2D spatial Fast Fourier Transform (FFT) to the extracted batch. This operation is applied in parallel to each frame of the batch with \textit{cuFFT} using \textit{XtPlanMany} data shaping. Applying the transformation to a batch of images allows to reduce the number of memory copies and increase the processing speed.\\

Then, the batch of frames is inserted into the time transformation queue. This data structure can contain a number of frames that is a multiple of batch size. Once the time transformation queue is full, the temporal transform of the batch of holograms is done by Principal Component Analysis (PCA), which is achieved in practice by data eigen-decomposition. In particular, the construction of a covariance matrix of a batch of holograms, its eigen-decomposition, and the projection of the result into a sub-basis are performed with \textit{cublasCgemm} and \textit{cusolverDnCheevd}. Then, data is copied to an accumulation buffer of the same size, and frames between two user-selected indices are flattened into a single image.

\subsection{Post-processing and output}
Post-processing operations such as image convolution, image re-normalisation, image accumulation, and contrast correction are sequentially applied to the frame. Once a frame has been post-processed it is copied to the output queue. This queue is used to circumvent latency and throughput issues during image recording. When a frame has been successfully rendered or recorded, a new frame is retrieved from the queue and copied to the CPU memory to be displayed or recorded.


The input thread and the compute thread are working simultaneously on the input queue. This implies that we need to take care of synchronisations to avoid a data race eventually leading to data corruption. A naive and inefficient approach would be to lock the entire queue using a mutual exclusion (mutex) rule when a thread reads from or writes to the queue. To have more flexibility, we designed a fine-grained method by dividing the input queue into several batches, each of them containing a fixed number of images. A mutex and a dedicated CUDA stream are assigned to each batch slot. CUDA streams allow to perform multiple read and write operations from and to different batches simultaneously on the GPU. Mutexes prevent the consumer thread to read data from a batch while the producer is writing on it and vice versa. This method reduces the number of GPU synchronisations to the minimum because each CUDA stream will handle its memory transfers sequentially without requiring explicit synchronisations in the source code. Finally, our approach eliminates the issue of conflicting memory accesses and increases greatly the input throughput.

\subsection{Loading and processing input images as batches}
Reading frames from a camera or a file is the main bottleneck as copies from the CPU memory to the GPU memory are longer than the processing time. Thus, we load multiple frames at once to reduce the number of system calls and reduce the latency. Then, frames are copied from host to device in batches of a specific size selected by the user. This method circumvents disk access issues and improves substantially the input throughput. Moreover, the entire input data is copied into GPU memory if enough memory is available on the device. Consequently, copies from the GPU file buffer to the GPU input queue are much faster since both memory areas are in the same address space (copy from device to device).

\subsection{Leveraging CUDA streams}
To improve both the processing time and the input throughput, we have to use the full potential of the GPU by maximising its workload. We noticed that some operations were not using all the Streaming Multiprocessors (SM) of the GPU. To alleviate this issue, we introduced different CUDA streams to execute more operations in parallel by exploiting all the SMs. In particular, it allows to perform computations and memory transfers in parallel as cells performing computations are different from cells handling data transfers. As opposed to previously, where all instructions were issued on the default stream, this change led to a significant performance improvement.

\subsection{Avoiding interruptions of high-speed cameras}
To read frames from high-speed cameras we rely on frame grabbers that can be configured to let the CPU acquire raw frames without having to process end-of-frame interrupts. These interruptions can become a bottleneck when dealing with very high frame rates. The program tracks the number of interrupts emitted by the camera since the last acquisition to determine the number of frames that are ready to be acquired. Images are extracted directly from the camera buffer and additional metadata is neither emitted nor read. Furthermore, we rely on a pinned memory buffer allocated on the host to allow fast copies from the CPU memory to the GPU memory. \\

Pinned memory is used as a staging area for transfers from the device to the host. It increases the data throughput and reduces potential lags by avoiding the cost of the transfer between page-able and pinned host arrays by directly allocating our host arrays in pinned memory. Allocates pinned host memory in CUDA C/C++ using \emph{cudaHostAlloc()} that allows use of flags, and de-allocates it with \emph{cudaFreeHost()}

\subsection{Limiting CPU-GPU synchronisations}
The CPU and the GPU are assigned to different tasks which are dissociated and run asynchronously. However, at some point, the host needs the result of the processing performed on the device. Thus, synchronisation is necessary as the CPU has to wait before retrieving data from the GPU. It is fundamental to correctly handle the synchronisation as too many synchronisations would have a negative impact on the performance and not enough synchronisations could corrupt the data. In our case, only the output rendering requires synchronisation between the CPU and the GPU. We reduced the number of calls to the \textit{cudaDeviceSynchronize} method to one at each output frame computed. The rest of the time both processing units run fully asynchronously.

\subsection{Handling non-square input frames}
Previously, non-square input images were padded with zeros so that the width matches the height of the frame. Therefore, rectangular images were handled similarly to square images to simplify computations. However, this naive approach has two disadvantages that limit the input throughput. Firstly, padding the input data is costly and memory intensive. Secondly, some parts of the input were irrelevant but were still being processed by our computation pipeline. Therefore, we adapted all operations performed on images and memory transfers to support non-square inputs. This change led to a substantial improvement in the processing time and the input throughput as using rectangular images is frequent in our setup.

\subsection{Minimising data allocations and transfers}
Handling memory efficiently is crucial when designing software based on an intensive GPU computation pipeline. We used NVIDIA Nsight Systems to analyse GPU memory utilisation and avoid memory re-allocations. As GPU memory allocation is very costly, we allocate all buffers only once when the program starts and free them when the program stops. Similarly, memory transfers between the CPU and the GPU or between buffers were reduced to a strict minimum to limit the number of calls to \textit{cudaMalloc} and \textit{cudaMemcpy}.

\subsection{Micro-cache}
The micro-cache feature represents a memory access optimisation bridging the two key components, the CPU and GPU. It streamlines (and optimises) the transmission of environment variables \textbf{(yet to be defined)} from the State \textbf{(yet to be defined)} to the Back end \textbf{(yet to be defined)}. Leveraging micro-cache yields a substantial boost in overall system performance by diminishing access times to shared data residing between the State and Back end. Without mu-cache, frequent intempestive data transfers between the pinned memory has to...\\

the static micro-cache is a low latency solution that circumvents the issue of the necessity of constant updates between the compute thread and the global state holder.\\

While Holovibes is in the process of computation, you have the capability to modify certain computation settings, such as SpaceTransformation functions or the z-distance parameter, among others. However, a significant challenge arises when attempting to change these variables, as they can only be altered once Holovibes completes all ongoing computations. Some variables, like the z-distance, are relatively straightforward to modify and can be changed almost anytime. On the other hand, variables like Batch-Size require more intricate operations, involving memory de-allocation and buffer reallocation. These operations cannot be performed until we have finished using the data stored in those buffers.

To address this challenge, we developed Micro-cache, which allows us to define "callback-like functions" for each type of variable. With this system, we can specify that the z-distance can be changed immediately, whereas for variables like batch size, which necessitate adjustments to buffer sizes, Micro-cache is designed to wait until all currently computed images have finished processing before applying the changes.

It's important to note that we are discussing the Micro-cache's pipeline here, distinct from the compute pipeline within Holovibes. The compute pipeline is essentially a list of functions that are applied to each batch of frames (with a size of Batch-Size).

For each variable, we determine the necessary actions by traversing a pipeline, which is described in detail below.

\subsection{Time Complexity}
