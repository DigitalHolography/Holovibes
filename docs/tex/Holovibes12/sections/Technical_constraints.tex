Optical Coherence Tomography (OCT) and Doppler-contrast imaging hinder real-time image rendering because of their high complexity. Coherent imaging applications thus require high camera acquisition rates.\\

The high-quality optical imaging of the eye fundus requires, for example, a frame sampling rate of 60,000 images per second. Coherent-light imaging of retinal layers typically needs 600 consecutive frames in less than 10ms. Beyond that duration, the phase of the back-scattered light is not sufficiently stable for wavelength-swept OCT \cite{hillmann2012common, spahr2019phase}. The frame rate requirement for high-quality Doppler imaging of retinal blood flow with a monochromatic light source is of the same order of magnitude \cite{Puyo2020Spatio}.\\

High-speed hologram rendering using a special-purpose calculation circuit printed on a Field-Programmable Gate Array (FPGA) enables efficient repetitive Fresnel diffraction calculations \cite{yamamoto2020special, hara2022design}. This FPGA approach is extremely efficient for ultrahigh-speed image rendering from small input frames. It is reported to be 23 times faster than with a GPU. It may be used for actual product development involving high-speed hologram rendering. Nevertheless, FPGA rendering cannot provide the computation customisation possibilities offered by general-purpose GPUs.\\

One of the most important technical issues for GPU hologram rendering is to enable the highest optical interferogram sampling rate in real-time with low latency. The latest generation of domestic computers come with commodity hardware that can allow the transfer of digitised interferograms from streaming cameras. The said transfer can then be done at a rate of several tens of thousands of images per second (This corresponds approximately to 4 GB/s), in parallel with data processing.\\

Our previous work (\textbf{source?}) put real-time digital holography with camera frame rates of the order of hundreds of frames per second \cite{Puyo2020Realtime} on display. The presence of a large footprint of spurious signals in ophthalmology requires high-quality results. The frame rate is still insufficient to provide these. To circumvent that issue, we leverage ultra-high-speed streaming cameras, reaching the required rate of 60,000 frames per second.