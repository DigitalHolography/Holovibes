CPUs are general-purpose processors designed to handle a wide range of tasks, from running the operating system and software applications to managing system resources. In contrast, GPUs are specialised processors initially developed for graphics rendering but now used for a specific subset of tasks that benefit from parallel processing, particularly mathematical calculations.\\

CPUs are optimised for sequential processing, executing instructions one after the other. GPUs, on the other hand, excel at parallel processing, where multiple calculations can be performed simultaneously, making them ideal for tasks that can be broken down into smaller, independent units of work.\\

CPUs are well-suited for tasks that require high single-threaded performance, such as everyday computing, web browsing, and tasks that rely heavily on the CPU's clock speed. GPUs shine in data-parallel workloads, such as 3D rendering, video encoding, scientific simulations, and machine learning, where they can handle massive amounts of data simultaneously.\\

CPUs are more versatile and can execute a wide variety of instructions, making them suitable for handling diverse workloads. GPUs are less flexible and are mainly designed for specific types of computations, which means they may not be as versatile as CPUs.\\

CPUs are typically designed to be power-efficient, making them suitable for laptops, smartphones, and other battery-powered devices. GPUs, due to their high core counts and power-hungry nature, consume more power and are often used in desktop computers or servers with adequate cooling and power supplies.\\

In summary, CPUs and GPUs complement each other in modern computing systems. CPUs are essential for general-purpose computing and handling tasks that require sequential processing, while GPUs excel in tasks that can be paralleled, particularly those involving graphics rendering, scientific simulations, and machine learning. Many modern applications and systems leverage both CPU and GPU resources to achieve optimal performance and efficiency.\\