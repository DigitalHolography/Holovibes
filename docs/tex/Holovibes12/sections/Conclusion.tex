We demonstrated that ultrahigh-speed digital hologram rendering from digitised interferograms can be performed in real-time with a commodity computer and GPU. Image rendering by propagation of the angular spectrum of 256-by-256-pixel interferograms, and temporal Fourier transformation of 512-frame batches in streaming was achieved at the rate of $\sim$ 60,000 frames per second (data rate of $\sim$4 GB/s), the highest throughput of an Ametek Phantom S710 camera, which is currently the fastest streaming camera of the state-of-the-art. This performance level is reached for uninterrupted streaming without any frame drop, and the maximum latency is about 30 ms.\\

This performance represents a significant improvement in the input throughput compared to our previous results \cite{samson2011video, Bencteux2015Holographic, Puyo2020Realtime}. It was made possible by leveraging the use of the GPU and maximising the workload with a series of software architecture changes and advanced optimisations. The holovibes software enables the highest digital hologram rendering bit-rate from uninterrupted frame streaming at high throughput from the state-of-the-art. It is a step toward ultra-high-speed rendering of digital holograms which is a requirement for coherent-light imaging applications, and will also pave the way for actual product development.\\

The current throughput bottleneck is host-to-device memory transfers of input frames, which indicates that even higher input throughput can be attained. Future software enhancements will include the implementation of a C++ Application Programming Interface (API) that will be created along with a Global State Holder (GSH) system that will gather and centralise all the global state of the program. Rules around the GSH will be set in order to only accept queries from the API, to facilitate integration of the rendering engine.\\

Furthermore, high levels of code factorisation, class abstraction (hiding all but relevant data on any object to reduce complexity and increase efficiency), encapsulation (a class should not have access to private data of another class) and inheritance (a parent class transfers all its attributes and methods to a child class) will be enforced in order to make the source code of the software as usable as possible and to facilitate its maintenance. The C++/CUDA code repository of the holovibes software will be made open source.\\

This work was supported by the French National research agency (ANR LIDARO and Laboratory of Excellence ANR-10-LABX-24), and the (\href{http://www.digitalholography.org}{digital holography foundation}).