\renewcommand{\arraystretch}{1.25}
\begin{table*}[t]
    \begin{ruledtabular}
        \begin{tabular}{lccccc}
            \textrm{Version}&
            \textrm{Input format}&
            \textrm{Input size}&
            \textrm{Input rate}&
            \textrm{Input throughput}& 
            \textrm{Output rate}\\
            \colrule
            Holovibes 8.0 & 16-bit & 1024$\times$1024 & 100 Hz &  33.5 GV/s & FPS \\
            \colrule
            Holovibes 9.0 & 16-bit & 1024$\times$1024 & 1,000 Hz & 33.5 GV/s & xx FPS \\
        \end{tabular}
    \end{ruledtabular}
    \caption{\label{tab:results_2} time : 32 frames. Performance comparison between Holovibes 8.0 and 9.0 when streaming from a file. The input file is composed of 16-bit 1024$\times$1024 frames that are read continuously. The output throughput is reported in terms of GigaVoxels (GV) per second. Each complex-valued voxel holds two single floats.
    }
\end{table*}

%The performance evaluation of our method was performed with a Ametek Phantom S710 CoaXPress streaming camera, four Bitflow Cyton-CXP CoaXPress frame grabbers, and an NVIDIA RTX Titan GPU with Cuda 11.1 on Microsoft Windows 10 64-bit operating system. Digital holograms were processed by the angular spectrum propagation method up to 60,000 frames per second for 256$\times$256 pixel frames and at 15,000 frames per second for 512$\times$512 pixel frames, followed by short-time Fourier transform temporal demodulation in time windows of 512 frames. Hologram rendering and visualization reached $\sim$117 frames per second and $\sim$29 frames per second, respectively. That accounts to an input throughput increase from 1.05 Gb/s to 3.93 Gb/s which is a $\sim$274 \% relative improvement. The performance level obtained while streaming from a camera before and after the optimizations by Holovibes 8.0 and Holovibes 9.0 respectively, are reported in Table \ref{tab:results}. This benchmark did not yield any experimental result because the Bitflow frame grabbers acquisition software was still under development at that time, and the frame readout from the S710 camera with was not yet properly implemented.\\

We assessed the performance level of hologram rendering with a Ametek Phantom S710 CoaXPress streaming camera, one Euresys Coaxlink Octo CoaXPress frame grabber, and an NVIDIA RTX Titan GPU with CUDA 11.1 on Microsoft Windows 10 64-bit operating system, in a laser Doppler optical setup \cite{Puyo2020Realtime}. Digital holograms were processed by the angular spectrum propagation method from 512$\times$512 pixel frames. In practice, we were able to perform real-time laser Doppler holography at 5,000 frames per second by short-time Fourier transform temporal demodulation in time windows of 128 frames, and image rendering at a rate of $\sim$ 39 Hz (\href{https://www.youtube.com/shorts/4-KwdM71jRQ}{video}). The performance level obtained while streaming from a camera before and after the optimisations by Holovibes 8.0 and Holovibes 9.0 respectively, are reported in Table \ref{tab:results}. This experimental demonstration was made possible by the implementation of the software optimisations described in this report.